
\documentclass[a4paper,11pt]{article}
\usepackage{mathrsfs,amssymb,stmaryrd,amsmath,mathtools,algorithm}
\usepackage{algorithmicx}
\usepackage{algpseudocode}
\usepackage{amsthm}
\usepackage{enumerate}
\usepackage{multirow} %multi row for tables
\usepackage{graphicx}
\usepackage{float}
\usepackage{xcolor}
\usepackage{verbatim}
\usepackage{tabularx}
\usepackage{longtable}
\usepackage[UTF8]{ctex}
%\usepackage{CJK}
\allowdisplaybreaks


\def\si{-\!\!\!*~}

\def\lra{\leftrightarrow}
\def\ra{\rightarrow}
\def\Lra{\Leftrightarrow}
\def\Ra{\Rightarrow}
\def\lb{\llbracket}
\def\rb{\rrbracket}

\def\dom{\mathrm{dom}}

\newtheorem {definition}{定义}
\newtheorem{theorem}{定理}[section]
\newtheorem{lemma}{引理}[section]


\title{Notes for Inira's SepLogic}
\author{Bowen Zhang}
\date{2020/04/05}

\begin{document}
	\maketitle
\section{Induction}
\subsection{approaches to programs verification}

	\subsubsection{程序验证}
	\begin{enumerate}
	\item specification用来描述程序要计算什么,而不是如何计算
	\item “一个程序是否满足一个specification”是不可判定的,因此验证工具需要能够引导验证的过程
	\item 程序验证的传统方法
		\begin{itemize}
		\item	\textbf{VGC——Verification Condition Generator}
		\\将程序进行拆解,对每个语句验证是否满足对应的spec,如果每个语句都可以满足对应的输入和输出,那么说明程序可以满足针对整体的spec。
		\item \textbf{Interactive Theorem Prover (proof assistant)}
		\\ 可以将“程序需要满足的spec”视为一个定理,并采用交互式定理证明器对其证明,但其中有两个难点:
			\begin{itemize}
			\item 将源语言进行编码,使其成为Programming Language
			\item 有关程序的推理,要建立在数学逻辑上
			\end{itemize}
		因此建立一个数学逻辑和PL之间的桥梁,是一个挑战.
		\end{itemize}
	\end{enumerate}
\subsubsection{交互式证明}
	旨在构造一个公理化的,对PL有逻辑上的语法和语义定义的,定理证明器。理论上,对于给定的程序,可以证明任意spec是否能满足。而实际中,由于程序语法的显式定义十分笨拙,因此很难对其进行操作。尽管如此,这种方式还是在底层程序的验证上扮演了十分重要的角色。
	
	人们也可以在定理证明器自带的逻辑系统中,开发编程语言(PL)。而浅度嵌入
(\textbf{shallow embedding approach})的理念,就是将\textbf{编程语言的程序}与\textbf{逻辑的程序}联系起来。这个理念实现起来有三种方式:
	\begin{itemize}
		\item 在逻辑系统中编写程序,将其转换为常规的PL
		\item 将源代码编写的程序,通过反编译,转换为逻辑程序
		\item 写一遍传统的PL,并写一遍逻辑程序
	\end{itemize}

	然而,所有基于\textbf{shallow embedding approach}的实现都面临着一个问题,那就是:解决PL和逻辑上的差异,尤其是对\textbf{偏函数和可修改状态}的处理。
	
	\begin{definition}
		\textbf{偏函数(partial function)}:一个偏函数$Pfun[A,B]$是一个一元函数,接收一个类型为$A$的参数$x$,返回类型为$B$的值,但$x$的选取可以不覆盖$A$的整个定义域。也就是说,偏函数只处理了定义域中的一个子集。
	\end{definition}
	该篇论文的方法是对给定的源代码,生成有关代码行为的逻辑命题。换句话说,就是对给定程序所生成的spec,为该spec生成可以使其满足的,一个充分的前提。该前提则是可以被交互证明的。\textbf{通过不显式地表达程序语法,可以避免深度嵌入的困难;而通过不依赖逻辑来表示程序,又可以避免浅度嵌入的困难。}
\subsubsection{特征公式}
	Characteristic formulae是一种关于程序行为的描述。它最初并应于与进程验算(一个并行计算的模型)。那是特征公式是采用时态逻辑,根据特定语法来生成描述进程的命题。它的最基本的结果就是:两个进程等价\textbf{iff}两个特征公式等价。因此以特征公式来证明两个进程之间的相等或不等关系。

	最近Honda等人将特征公式由进程逻辑转向了程序逻辑,以PCF\\
(Programming Language for Computable Function)为研究目标,PCF可以被理解为简化版的ML(如Caml或SML)。Honda等人给出了一个生成给定PCF程序的“total characteristic assertion pair”的算法,这个“完全特征断言对”就是该程序的最弱前置条件,和最强后置条件。注意,PCF程序并没有spec作为注解,也没有循环不变量,因此最弱前置条件的算法与常规有差异。而有关一个程序的普适spec的概念更为古老,起源于Hoare逻辑的完全性证明。
	
	Honda工作的创新之处就在于,一个PCF程序的普适spec可以不依赖编程语言的语法来表示。Honda等人表明,特征公式可以通过证明“普适的spec 逻辑蕴含 目标的spec”,来证明一个给定的程序满足其对应的spec。因此证明过程可以在不需要引用变成语言的语法的情况下进行。但这种idea面临着一个主要的问题,就是“完全特征断言对”的spec语言,是由专门的逻辑系统表示的,其中变量代表了PCF的值(包括了非终止函数),也因此等价被解释成为观测意义上的等价,即对比变量是否相同。由于这种特殊的逻辑系统没法轻易编码,转换为标准逻辑系统。并且实现一个新逻辑系统之上的定理证明器需要投入大量的时间。因此Honda等人的工作只停留在理论层面,并未产生一个有效的程序验证工具。

	该论文的工作重新发掘了特征公式,同时寻找到一种方法来增强嵌入的深度。在这种方法中,程序语法被显式地表示在定理证明程序中。\textbf{该论文通过构建逻辑公式,来捕获能够进行深度嵌入的推理,并且不需要在任何时候以程序语法表示}。从某种意义上,这种方法可以被理解为在深度嵌入之上构建了一个抽象层,隐藏掉技术细节却保留了优点。与Honda等人不同,该论文建立的特征公式是用标准高阶逻辑表示的,也因此能通过特征公式建立一个实用的程序验证工具。

\subsubsection{特征公式}
	一个项$t$(term $t$)的特征公式被记为$\llbracket t \rrbracket$。特征公式与Hoare三元组之间有很深的联系。Hoare三元组$\left \{H \right \} t \left \{Q \right \}$(这里即分离逻辑中的Hoare三元组)断言了:如果一个堆可以满足谓词$H$时,执行项$t$终止后返回值$v$,那么更新后的堆可以使得$(Q~v)$满足。注意后置条件$Q$需要输出值和输出堆同时满足。当$t$的类型为$\tau$时,前置条件$H$具有类型$Heap \to Prop$,后置条件$Q$的类型则为$\left \langle \tau \right \rangle \to Heap \to Prop$,其中$Heap$是堆的一个类型,而$\left \langle \tau \right \rangle$是Coq中的类型,与ML中的$\tau$类型相对应。

特征公式$\llbracket t \rrbracket$是使得$\llbracket t \rrbracket~H~Q$与$\left \{H \right \} t \left \{Q \right \}$逻辑等价的谓词,但是特征公式与Hoare三元组具有本质上的区别:三元组$\left \{H \right \} t \left \{Q \right \}$是一个三元关系,它的第二个参数需要调用语法来描述项$t$;相对的,$\llbracket t \rrbracket~H~Q$是一个逻辑命题,它以标准的高阶逻辑连接词表示,如$\wedge ,\exists ,\forall$和$\Rightarrow$,并不依赖于项$t$的语法。而且Hoare三元组的推导,需要建立在Hoare逻辑特有的推导规则上,然而特征公式只需要运用基本的高阶逻辑就可以进行推理,不需要额外引入推导规则。

本章接下来部分,该论文展示了构造特征公式的核心思想,重点讨论了\textbf{let-binding}的处理,包括了函数的应用以及定义应用,并解释了如何处理可以实现局部推理的Frame Rule。

\subsubsection{Let-binding}
为了评估一个形如"\textbf{let}	$x=t_1$ \textbf{in} $t_2$"的项,我们第一步需要分析它的子项$t_1$,接下来带着输出的结果再分析$t_2$。为证明表达式能接受$H$作为前置,$Q$为后置条件。我们需找到$t_1$的一个有效的后置条件$Q'$,该后置条件描述了$t_1$执行后和$t_2$执行前的内存状态,并能接受由$t_1$所生成的结果$x$。因此,可以使用($Q'~x$)来表示$t_2$的前置条件。下面是上述内容对应的Hoare规则:
	\begin{equation*}
	\begin{tabular}{c}
	$\left \{ H \right \} t_1 \left \{ Q \right \} \quad \forall x. \left \{ Q' x \right \} t_2 \left \{ Q \right \} $
	\\
	\hline
	$\left \{ H \right \} \left (\mathbf{let} ~t_1~\mathbf{in} ~t_2 \right ) \left \{ Q \right \}$
	\end{tabular}
	\end{equation*}
而特征公式对let-binding的构建如下:
\begin{equation*}
\llbracket \mathbf{let} ~x=t_1~\mathbf{in} ~t_2 \rrbracket~\equiv~\lambda H.\lambda Q.~\exists Q'.~\llbracket t_1 \rrbracket~H~Q'~\wedge~\forall x.~ \llbracket t_2 \rrbracket~(Q'~x )~Q  
\end{equation*}
这个公式与Hoare规则很相似,唯一不同之处在于:在特征公式中,中间的后置条件$Q'$是明确以存在量词来引入的,而这个量词在Hoare逻辑的规则中是被隐式掉的。之所以能够对未知spec做量化,是建立在高阶逻辑的特性上的,这个特性在该论文的工作中起到了很关键的作用。这种方法与传统的方法形成了鲜明的对比,后者必须在验证源程序时,体现出中间的spec,甚至是循环不变量。

为了使证明更加易读,作者介绍了一个关于特征公式的符号系统。比如对于let-binding,他定义了
\begin{equation*}
( \mathbf{let} ~x=\mathcal{F}_1~\mathbf{in} ~\mathcal{F}_2 ) ~\equiv~\lambda H.\lambda Q.~\exists Q'.~\mathcal{F}_1~H~Q'~\wedge~\forall x. ~\mathcal{F}_2~(Q'~x)~Q  
\end{equation*}
粗体的关键词对应了逻辑公式中的符号,如($\mathbf{let} ~...~\mathbf{in}...$),而非粗体则对应于编程语言语法中的实际构造,如($x=\mathcal{F}_1$ 或 $\mathcal{F}_2 $)。特征公式的生成,就可以归结为对编程语言关键词的重新解释。

这样做的结论就是,特征公式可以像源语言一样,准确且优美地进行表达。用“$\llbracket t \rrbracket~H~Q$”就可以直接给人们呈现,由源代码$t$后面接着前/后置断言的表示方法。注意这种表达方式可以不仅应用于顶层的项$t$,还可以在证明$t$正确性的过程中,应用于所有$t$的子项。

\subsubsection{Frame Rule}
	“局部推导(local reasoning)”指的是能够只验证与内存相关的部分代码的推理方式,它涉及了代码的执行。运用局部推导时,所有没被提及的内存单元都会被默认为保持不变。而局部推导的理念,在分离逻辑中的Frame rule得到了优雅的表示。Frame rule表明了如果程序将一个由谓词$H_1$描述的完整的堆,转移到了由谓词$H_1'$描述的堆,那么对于任意的堆谓词$H_2$,这个程序同样也可以将形如$H_1*H_2$的堆,转移为$H_1'*H_2$的堆。

比如将内容加一函数\textbf{incr},应用于内存中的地址$l$。假设($l \hookrightarrow n$)描述了单堆,那么函数\textbf{incr}的应用使一个形如($l \hookrightarrow n$)的堆转换为($l \hookrightarrow n+1$)的堆。而通过frame rule,人们可以将函数\textbf{incr}的推导,扩大为由一个形如($l \hookrightarrow n$)$*$($l' \hookrightarrow n'$)的堆,转移至($l \hookrightarrow n+1$)$*$($l' \hookrightarrow n'$)的堆,其中分离合取则断定了$l$与$l'$之间互不相等。分离合取的运用,为我们提供了描述除$l$内容加一之外,系统其它属性的可能。

Frame rule以Hoare三元组的形式定义如下:
	\begin{equation*}
	\begin{tabular}{c}
	$\left \{ H_1 \right \} t \left \{ Q_1 \right \}$
	\\
	\hline
	$\left \{ H_1*H_2 \right \} t \left \{ Q_1 \star H_2 \right \}$
	\end{tabular}
	\end{equation*}
其中,($\star$)和($*$)的区别在于后置条件中,$\left \{ Q_1 \star H_2 \right \}$被用来描述$\lambda x.~(Q_1~x)*H_2$,这里$x$描述了输出的值,$(Q_1~x$描述了输出的堆。

为了将frame rule和特征公式联系在一起,作者引入了一个谓词\textbf{frame}(即为课件中的\textbf{local}),该谓词的定义为:要证明命题“\textbf{frame} $\llbracket t \rrbracket~H~Q$”,可以通过将$H$拆解为$H_1*H_2$,将$Q$拆解为$Q_1 \star H_2$,进而证明$\llbracket t \rrbracket~H_1~Q_1$成立,其形式化的定义如下:
\begin{equation*}
\mathbf{frame}~\mathcal{F} ~\equiv~\lambda HQ.~\exists H_1 H_2 Q_1.\begin{cases}
H = H_1 * H_2
\\ \mathcal{F}~H_1~Q_1
\\ Q = Q_1 \star H_2
\end{cases} 
\end{equation*}

frame rule并不能支持语法推导,也就是说当需要运用该规则时,不能只通过$t$的形式来猜测。然而作者想通过系统化的方式,直接从源语言得出它的语法表示,进而生成特征公式。但因为推理时在哪插入\textbf{frame}谓词是不固定的,因此作者在每一个特征公式上都加入了\textbf{frame}谓词。比如之前定义的let-binding就更新为:
\begin{equation*}
( \mathbf{let} ~x=\mathcal{F}_1~\mathbf{in} ~\mathcal{F}_2 ) ~\equiv~\mathbf{frame}~(~\lambda H.\lambda Q.~\exists Q'.~\mathcal{F}_1~H~Q'~\wedge~\forall x. ~\mathcal{F}_2~(Q'~x )~Q~) 
\end{equation*}

这个带有侵略性的策略,使得我们在推理的任何时候都可以运用frame rule。而如果不需要frame rule的话,它也可以直接被化简掉。$\mathcal{F}~H~Q$永远都是\textbf{frame}$\mathcal{F}~H~Q$证明时的一个充分的前提。

作者描述的用来处理frame rule的谓词,实际上是广义的。它同时可以在处理复合语句、增强前置条件、弱化后置条件,以及为模拟垃圾回收而允许的内存单元丢弃。


%$\frac{\left \{ H \right \} t_1 \left \{ Q \right \} ~~\forall x. \left \{ Q' x \right \} t_2 \left \{ Q \right \}}
%{\left \{ H \right \} t_1 \left \{ Q \right \}}$

\subsubsection{对类型的解释}
高阶逻辑能自然地描述基础值(如纯函数化的列表),甚至链表的数据结构在逻辑上的描述和在PL中的能够完美匹配。然而验证和推理程序的函数时,仍需要额外注意。甚至,PL中的函数,并不能直接像逻辑中的函数那样进行直接的表示,究其原因是因为两者的差异:PL中的函数可能出现分歧(字符串类型的函数引入整形参量)或者崩溃(引用null指针),而逻辑中的函数永远都会终止。为了解决这个差异,作者引入了一个新的数据类型\textbf{Func}来表示函数。\textbf{Func}这个类型在特征公式中被描述为一个抽象的数据类型。在可靠性证明中,一个类型为\textbf{Func}的值被解释为源语言中函数对应的语法表示。

另一个是对指针的特殊操作,也就是将Caml中的值对应到Coq中的值。当运用特征公式进行推理时,每个内存空间的类型和内容都通过堆谓词被显式地进行描述。因此就不需要指针携带着所指向内存单元的类型。也因此所有的指针在逻辑中,就都可以通过一个抽象的数据类型来描述,作者将这种类型定义为\textbf{Loc}。在可靠性证明中,一个\textbf{Loc}类型的值被解释为一个存储地址。

将Caml中的类型形式化转换为Coq中的类型,是通过一个操作符$\left \langle \cdot \right \rangle$,它将多有的箭头类型指向\textbf{Func}类型,并将所有的引用类型映射到\textbf{Loc}类型。一个Caml中类型为$\tau$的值,就可以被表示成Coq中类型为$\left \langle \tau \right \rangle$的值,操作符$\left \langle \cdot \right \rangle$的定义如下:
~\\
~\\
~\\~\\
\begin{align*}
	\langle int \rangle \quad & \equiv \quad \mathbf{Int}\\
  \langle \tau_1 \times \tau_2 \rangle \quad & \equiv \quad \langle \tau_1  \rangle \times  \langle \tau_2 \rangle \\
	\langle \tau_1 + \tau_2 \rangle \quad & \equiv \quad \langle \tau_1  \rangle + \langle \tau_2 \rangle \\
	\langle \tau_1 \to \tau_2 \rangle \quad & \equiv \quad \mathbf{Func}\\
	\langle \mathbf{ref} \tau \rangle \quad & \equiv \quad \mathbf{Loc}
\end{align*}

一方面,ML的类型系统对于将Caml的值直接映射为Coq中的值非常有帮助;另一方面,这种类型定义也是带有限制的:有很大数量的程序是正确的,但无法在ML中进行验证。特别是ML的类型系统不支持~空指针调用~和~强制类型转换~的程序。这些问题即使在无视类型安全的情况下,依旧很难处理。然而它们的正确性可以运用程序的正确性验证来证明。因此,作者对Caml引入了空指针和强制类型转换。并运用特征公式验证空指针永远不会被调用,以及从内存中读取的数据永远是预期的类型。

然而,这个方法的正确性并不能被直接且完整地验证。一方面,特征公式是由类型程序生成的;另一方面,空指针和强制类型转换有可能使类型推导出现问题。为了验证特征公式的可靠性,作者引入了一个新的类型系统\textbf{weak-ML}。这个类型系统并不具有可靠性,但是它能够携带用来生成特征公式的所有类型和信息和不变量,并证明它们是合理的。

简单来讲,weak-ML对应着一个更为松散的ML版本,即:不追踪指针或函数的类型,不对释放和应用函数的类型施加任何约束。因此,将Caml类型解释为Coq类型实际上需要两步:将Caml类型转译为weak-ML类型,将这个weak-ML类型的值转换为Coq类型。









\end{document}
